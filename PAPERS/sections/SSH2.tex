\section{Back to the SSH model}
\subsection*{Dispersion:}
Now if we compare our $k$-space SSH Hamiltonian in 
\eref{eq:Hk:SSH:compact}, we notice 
$h_0$ and $\vec{h}(k) = \vec{d}(k)$. Therefore
from \eref{eq:dispersion:tls}
\blgn
E_k
&=\pm |\dv(k)|=\pm\sqrt{(t_+ + t_-\cos ka)^2+t_-^2\sin^2 ka}\label{eq:disp:SSH:1}\\
&=\pm \sqrt{(t_+ - t_-)^2+4t_+t_-\cos^2(ka/2)}\non\\
&=\pm \sqrt{(4 \del^2+4t_+t_-\cos^2(ka/2)}\,.
\label{eq:disp:SSH:2}
\elgn
with 
$d_x(k)\equiv\tp+\tm\cos ka$, $d_y(k)\equiv \tm \sin ka$.

\begin{comment}
\begin{figure}[!htp]
\centering
\includegraphics[height=1cm,clip]{\NHHMDIR/SSH_chain_Asboth.png}
%\caption{Su-Schrieffer-Heeger (SSH) chains in two different confi.}
\end{figure}
\end{comment}

\subsection*{Other Nomenclatures:}
In many literatures, the intracell hopping $\tp = t+\del$
is denoted by $v$ and intercell hopping $\tm = t-\del$
is denoted by $w$. Now onward, we shall follow this notation
and according to it,
\blgn
d_x(k) &\equiv v+w\cos ka; \quad d_y(k)\equiv w \sin ka\,
\label{eq:dx:dy:in:v:w}
\elgn
and
\blgn
E(k)=\pm\sqrt{v^2+w^2+2vw\cos ka}\,.\non
\elgn

From \eref{eq:dx:dy:in:v:w},
we get
\blgn
(d_x-v)^2 + d_y^2 &= w^2
\elgn
which describes a circle of radius $w$, centered at $(v,0)$.

%
% \begin{figure}[!htp]:
% \centering
% \includegraphics[height=5cm,clip]{\NHHMDIR/SSH_dispersion_winding_no.png}
% %\caption{Su-Schrieffer-Heeger (SSH) chains in two different confi.}
% \end{figure}
\blfootnote{Courtesy: Asb\'oth \etal, \url{arXiv:1509.02295v1} }
